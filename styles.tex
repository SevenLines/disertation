%%% Макет страницы %%%
\geometry{a4paper,top=2cm,bottom=2cm,left=2.5cm,right=1cm}

%%% Кодировки и шрифты %%%
\renewcommand{\rmdefault}{ftm} % Включаем Times New Roman

%% полуторный интервал

%%% Выравнивание и переносы %%%
\sloppy					% Избавляемся от переполнений
\clubpenalty=10000		% Запрещаем разрыв страницы после первой строки абзаца
\widowpenalty=10000		% Запрещаем разрыв страницы после последней строки абзаца
\linespread{1.5}

%%% Библиография %%%
\makeatletter
\bibliographystyle{utf8gost705u}	% Оформляем библиографию в соответствии с ГОСТ 7.0.5
\renewcommand{\@biblabel}[1]{#1.}	% Заменяем библиографию с квадратных скобок на точку:
\makeatother

%%% Изображения %%%
\graphicspath{{images/}} % Пути к изображениям

%%% Цвета гиперссылок %%%
\definecolor{linkcolor}{rgb}{0.9,0,0}
\definecolor{citecolor}{rgb}{0,0.6,0}
\definecolor{urlcolor}{rgb}{0,0,1}
\hypersetup{
    colorlinks, linkcolor={linkcolor},
    citecolor={citecolor}, urlcolor={urlcolor}
}

%%% Оглавление %%%
\renewcommand{\cftchapdotsep}{\cftdotsep}

%%% Математические высказывания %%%
\newtheorem{definition}{Определение} \numberwithin{definition}{section}
\newtheorem{theorem}{Теорема} \numberwithin{theorem}{section}
\newtheorem*{theorem*}{Теорема}
\newtheorem{lemma}{Лемма} \numberwithin{lemma}{section}
\newtheorem*{lemma*}{Лемма} 
\newtheorem{state}{Утверждение} \numberwithin{state}{section}
\newtheorem{proposition}{Предложение} \numberwithin{proposition}{section}
\newtheorem{corollary}{Следствие} \numberwithin{corollary}{section}
\newtheorem{remark}{Замечание} \numberwithin{remark}{chapter}      

\chapter{Контурные изображения}
\section{Базовые понятия}
Растровое изображение есть функция $I_rgb(x,y): N \times N \to([0,255],[0,255],[0,255])$. Таким образом каждой точке $(x,y)$ мы сопоставляем тройку $(r,g,b)$. Первый элемент тройки соответствует красной компоненте цвета в растровом изображении, второй -- зеленой и третий -- синей. Далее будем использовать следующую запись:

$$
\begin{array}{l}
I_r(x,y) = I_{rgb}(x,y)_r \\
I_g(x,y) = I_{rgb}(x,y)_g \\
I_b(x,y) = I_{rgb}(x,y)_b \\
\end{array}
$$

Определим изображение заданного в оттенках серого
$I_grey(x,y): N \times N \to([0,255])$
Введем оператор <<обесцвечивания>> $D$, который позволяет перейти от цветного изображения $I_rgb$ к изображению $I_grey$ заданному в оттенках серого:
D($I_rgb$) = $I_grey$
Существует несколько основных способов обесцвечивания изображения:
\begin{enumerate}
 \item Красный канал $D_{red}(I_{rgb}) = I_r$
 \item Зеленый канал $D_{greeb}(I_{rgb}) = I_g$
 \item Синий канал $D_{blue}(I_{rgb}) = I_b$  
 \item Среднее значение (average): $D_{avg}(I_{rgb}) = \frac{I_r + I_g + I_b}{3}$
 \item Лума (luma), учитывает особенности восприятия цвета человеком: $$D_{luma}(I_rgb) = I_r\cdot0.3 + I_g\cdot0.59 + I_b\cdot0.11$$
 		Значения коэффициентов, иногда, могут отличаться от приведенных выше, но их сумма всегда равна 1
 \item Минимум  $D_{min}(I_{rgb}) = min(I_r, I_g, I_b)$
 \item Максимум  $D_{min}(I_{rgb}) = max(I_r, I_g, I_b)$
 \item Обесцвечивание (desaturtaion): $D_{desaturation}(I_{rgb}) = \frac{D_{min}(I_{rgb}) + D_{max}(I_{rgb})}{2}$
\end{enumerate}

Наилучший результат для средне-статистического изображения (с нормальной гистограммой) получается при использование 5-го и последнего способов. Под наилучшим результатом понимается сохранение яркости (компоненты value в модели HSV) цветов исходного изображения.

Монохромное изображение есть функция $I(x,y): N \times N \to \{0,1\}$
Можно явно определить переход от изображения заданного в оттенках серого к монохромному изображению если задать некоторый порог $t\in [0,255]$ и ввести оператор отсечения (thresholding) $T_t$.
$$
I = T_t(I_{grey}) = \left\{ 
\begin{array}{ll}
0 & ,I_{grey} < t \\
1 & ,\text{иначе} 
\end{array}
\right\}
$$


\section{Формализация описания плоских контурных изображений}
Составляющими элементами плоских контурных изображений будем считать дуги и связи дуг. Дуга arc основной количественной характеристикой имеет сектор окружности, измеряемый в градусах (точнее, в количестве минимальных шагов возрастания градусной меры дуги, что обеспечивает конечность количественных характеристик в некоторой шкале или масштабе).

Отметим, что любые две несовпадающие точки $a$~и~$b$ на плоскости (задающие луч $\overline{ab}$)
можно соединить дугой заданной градусной меры $\alpha$ ($0\le\alpha\le 360$) ровно двумя способами, в первом
случае все точки дуги будут лежать справа от луча $\overline{ab}$, будем говорить что дуга обходится по часовой стрелке, во втором случае все точки дуги будут лежать слева от луча и речь будет идти об обходе против часовой стрелки. Для дуг градусной меры $\alpha \in \{0,360\}$ направление обхода не определено.

Связь дуг $rel$ основной количественной характеристикой имеет угол между дугами, измеряемый в градусах (точнее, в количестве минимальных шагов возрастания углов, что обеспечивает конечность количественных характеристик в некоторой шкале или масштабе).

Основными математическими моделями для данного подхода будут трехосновные алгебраические системы~\cite{7}\cite{8} вида
\begin{equation}
M = < Arc, Rel, V; Sector, Angle, R >
\end{equation}
где основное множество $Arc$ – совокупность дуг; основное множество $Rel$ – совокупность связей дуг; основное множество $V$ – некоторый начальный отрезок натуральных чисел (представляет сектора дуг и углы связей дуг в некоторой шкале); одноместная функция $Sector: Arc \rightarrow V$, т.е определяет градусную меру дуги; одноместная функция $Angle: Rel \rightarrow V$, т.е определяет угол связи дуг; трехместное отношение $R$ соединяет связь дуг $rel$ с соответствующими дугами, т.е. $R$ - подмножество декартова произведения 
$Rel \times Arc \times Arc.$

Для наших целей важно всегда работать только с конечными множествами, что достигается рассмотрением конечных множеств $Arc$, $Rel$, а также предположением о наличии минимального шага возрастания количественных характеристик дуг и связей дуг, т.е. конечное множество V имеет минимальное ненулевое значение, соответствующее минимальному шагу, и максимальное, соответствующее 360 градусам.
%\chapter{Оформление различных элементов} \label{chapt1}
%
%\section{Форматирование текста} \label{sect1_1}
%
%Мы можем сделать \textbf{жирный текст} и \textit{курсив}.
%
%%\newpage
%%============================================================================================================================
%
%\section{Ссылки} \label{sect1_2}
%Сошлёмся на библиографию: \cite{bib1}, \cite{bib2}, \cite{bib3,bib4,bib5}.
%
%Сошлёмся на приложения: Приложение \ref{AppendixA}, Приложение \ref{AppendixB2}.
%
%Сошлёмся на формулу: формула (\ref{eq:equation1}).
%
%Сошлёмся на изображение: рисунок \ref{img:knuth}.
%
%%\newpage
%%============================================================================================================================
%
%\section{Формулы} \label{sect1_3}
%
%\subsection{Ненумерованные одиночные формулы} \label{subsect1_3_1}
%
%Вот так может выглядеть формула, которую необходимо вставить в строку по тексту: $x \approx \sin x$ при $x \to 0$.
%
%А вот так выглядит ненумерованая отдельностоящая формула c подстрочными и надстрочными индексами:
%$$
%(x_1+x_2)^2 = x_1^2 + 2 x_1 x_2 + x_2^2
%$$
%
%При использовании дробей формулы могут получаться очень высокие:
%$$
%  \frac{1}{\sqrt(2)+
%  \displaystyle\frac{1}{\sqrt{2}+
%  \displaystyle\frac{1}{\sqrt{2}+\cdots}}}
%$$
%
%В формулах можно использовать греческие буквы:
%$$
%\alpha\beta\gamma\delta\epsilon\varepsilon\zeta\eta\theta\vartheta\iota\kappa\lambda\\mu\nu\xi\pi\varpi\rho\varrho\sigma\varsigma\tau\upsilon\phi\varphi\chi\psi\omega\Gamma\Delta\Theta\Lambda\Xi\Pi\Sigma\Upsilon\Phi\Psi\Omega
%$$
%
%%\newpage
%%============================================================================================================================
%
%\subsection{Ненумерованные многострочные формулы} \label{subsect1_3_2}
%
%Вот так можно написать две формулы, не нумеруя их, чтобы знаки равно были строго друг под другом:
%\begin{eqnarray}
%  f_W & = & \min \left( 1, \max \left( 0, \frac{W_{soil} / W_{max}}{W_{crit}} \right)  \right), \nonumber \\
%  f_T & = & \min \left( 1, \max \left( 0, \frac{T_s / T_{melt}}{T_{crit}} \right)  \right), \nonumber
%\end{eqnarray}
%
%Можно использовать разные математические алфавиты:
%\begin{eqnarray}
%\mathcal{ABCDEFGHIJKLMNOPQRSTUVWXYZ} \nonumber \\
%\mathfrak{ABCDEFGHIJKLMNOPQRSTUVWXYZ} \nonumber \\
%\mathbb{ABCDEFGHIJKLMNOPQRSTUVWXYZ} \nonumber
%\end{eqnarray}
%
%Посмотрим на систему уравнений на примере аттрактора Лоренца:
%
%$$
%\left\{
%  \begin{array}{rl}
%    \dot x = & \sigma (y-x) \\
%    \dot y = & x (r - z) - y \\
%    \dot z = & xy - bz
%  \end{array}
%\right.
%$$
%
%А для вёрстки матриц удобно использовать многоточия:
%$$
%\left(
%  \begin{array}{ccc}
%  	a_{11} & \ldots & a_{1n} \\
%  	\vdots & \ddots & \vdots \\
%  	a_{n1} & \ldots & a_{nn} \\
%  \end{array}
%\right)
%$$
%
%
%%\newpage
%%============================================================================================================================
%\subsection{Нумерованные формулы} \label{subsect1_3_3}
%
%А вот так пишется нумерованая формула:
%\begin{equation}
%  \label{eq:equation1}
%  e = \lim_{n \to \infty} \left( 1+\frac{1}{n} \right) ^n
%\end{equation}
%
%Нумерованых формул может быть несколько:
%\begin{equation}
%  \label{eq:equation2}
%  \lim_{n \to \infty} \sum_{k=1}^n \frac{1}{k^2} = \frac{\pi^2}{6}
%\end{equation}
%
%В последствии на формулы (\ref{eq:equation1}) и (\ref{eq:equation2}) можно ссылаться.
%
%%\newpage
%%============================================================================================================================
%
%\clearpage
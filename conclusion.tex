\chapter*{Заключение}						% Заголовок
\addcontentsline{toc}{chapter}{Заключение}	% Добавляем его в оглавление

%Основные результаты работы заключаются в следующем.
%\begin{enumerate}
%  \item На основе анализа \ldots
%  \item Численные исследования показали, что \ldots
%  \item Математическое моделирование показало \ldots
%  \item Для выполнения поставленных задач был создан \ldots
%\end{enumerate}
%И какая-нибудь заключающая фраза.
Основные результаты работы заключаются в следующем. 
На качественном уровне: для данных, организованных более сложно, чем  таблицы реляционных баз данных, доказаны очень близкие результаты по вычислительной сложности их анализа.
Более точно, основные результаты работы могут быть сформулированы следующим образом.
\begin{enumerate}
\item На основе анализа плоских контурных изображений разработана формализация, представленная ориентированными дугами, связями дуг и их численными характеристиками в градусном измерении и  относительными размерами  длины дуг.
\item Исследования показали, что вычислительная сложность  анализа плоских контурных изображений, представленных ориентированными дугами, связями дуг и их численными характеристиками в градусном измерении и  относительными размерами  длины дуг, почти не зависит от количества образцов.
%\item Математическое моделирование показало, что использование масштабных рядов контурных изображений уменьшает вычислительную сложность задачи поиска изоморфных вложений образцов в исследуемое изображение.
\item Для подтверждения теоретических результатов на практике  было созданы комплексы программ для решения задач.
\end{enumerate}

Полученные в диссертационной работе результаты соответствуют общей
тенденции развития математических методов решения информационных задач для больших массивов данных.

\clearpage
\subsection*{\Large Общая характеристика работы}
\fontsize{14pt}{15pt}\selectfont
\noindent
\underline{\textbf{Актуальность темы.}}\\
Разработка эффективных вычислительных математических методов решения информационных задач является важнейшим направлением развития современных информационных технологий (и в целом, прогресса, так как невозможно представить современную науку и технику  без использования компьютеров). 

В идеале, эти методы должны обеспечивать скорость решения ряда важных информационных задач вне зависимости от объема данных. И действительно, есть ряд важных  информационных задач, где это возможно. 

Наглядным (и очень важным) примером  этого являются реляционные базы данных (БД), где вычислимость запросов определенных типов не зависит от объема данных, а только линейно от сложности проекта самой БД.

Практическим подтверждением этого для обывателя является скорость работы банковских систем, использующих сетевые реляционные БД,  с их мировыми сетями терминалов и банкоматов, где можно проводить операции с вкладами и денежными средствами в любой точке мира за считанные секунды.

Отметим, что весьма близко к этому классу  примыкают задачи поиска данных по ключевым словам в Интернет - пространстве (полно текстовые поисковые системы «Google» [2], «Яндекс» и др.), где скорость поиска не замедляется из-за экспоненциального роста информации в глобальной сети.

Другим  важным классом информационных задач являются вопросы распознавания  образцов (образов), где при организации данных близкой к таблицам реляционных БД также могут быть получены результаты независимости скорости  распознавания  образцов от их количества [3], вернее, верхней границы сложности распознавания одного образца с добавкой только количества образцов. 

Данное направление исследований (разработка эффективных вычислительных математических методов решения информационных задач) весьма актуально для современной математики, науки в целом и техники, как в теоретическом, так и  практическом плане, где работают крупные транснациональные компьютерные корпорации, реализуются технологии «Big Table»,  «Big Data» и предполагается получение прорывных результатов в робототехнике, молекулярной биологии, системах искусственного интеллекта и других важных областях, имеющих определяющее значение для прогресса современного общества.


\underline{\textbf{Целью}} данной работы является построение математических моделей контурных изображений, которые, хотя и имеют более сложную организацию данных, чем  таблицы реляционных БД, но позволяют получить почти аналогичные результаты по вычислительной сложности  анализа контурных изображений, включая проверку изоморфных вложений образцов в анализируемое изображение.


Для~достижения поставленной цели необходимо было решить следующие \underline{\textbf{задачи}}:
\begin{enumerate}
	\item Разработать преобразование растра контурного изображения в нагруженный граф специального вида.
	\item Разработать преобразование нагруженных графов специального вида в математические модели, представленные многоосновными алгебраическими системами, где контурные изображения сведены к ориентированным дугам, связям дуг и их численным характеристикам в градусном измерении и  относительных размеров  длины дуг.
	\item Исследовать алгоритмическую сложность  анализа контурных изображений, включая проверку изоморфных вложений образцов в анализируемое изображение, где контурные изображения сведены к ориентированным дугам, связям дуг и их численным характеристикам в градусном измерении и  относительных размеров  длины дуг.
	\item Разработать  масштабные ряды контурных изображений и процедуры сжатия на  основе относительных размеров дуг.
	\item Исследовать возможность использования  изоморфного вложения сжатого образца в сжатое изображение для уменьшения алгоритмической сложности построения  изоморфного вложения исходного образца в исходное изображение.
	\item Исследовать возможность использования полученных математических методов, математических моделей представления данных, алгоритмов и комплексов программ для решения прикладных задач:
	\begin{enumerate}
		\item распознавания символов;
		\item оценки устойчивости битумных эмульсий;
		\item автоматизации составления проектов организации дорожного движения (ПОДД). 
	\end{enumerate}
\end{enumerate}

\noindent
\underline{\textbf{Основные положения, выносимые на~защиту:}}
\begin{enumerate}
	\item Сходящийся алгоритм волновой скелетизации,  обеспечивающий преобразование растра контурного изображения в нагруженный граф специального вида (Утверждение 2.3.1.).
	\item Оценка нижней границы алгоритмической сложности  анализа контурных изображений, включая проверку изоморфных вложений образцов в анализируемое изображение, где контурные изображения сведены к ориентированным дугам, связям дуг и их численным характеристикам в градусном измерении и  относительным размерам  длины дуг  (Теорема 2.5.1.).
	\item Необходимое и достаточное условие продолжения изоморфного вложения сжатого образца в сжатое изображение до изоморфного вложения исходного образца в исходное изображение  (Теорема 1.).
	\item Верхняя граница алгоритмической сложности построения продолжения изоморфного вложения сжатого образца в сжатое изображение до изоморфного вложения исходного образца в исходное изображение  (Теорема 2.).
	\item Уменьшение верхней границы алгоритмической сложности построения  изоморфного вложения исходного образца в исходное изображение при использовании изоморфного вложения сжатого образца в сжатое изображение  (Теорема 3.).
	\item Комплексы программ  решения прикладных задач:
	\begin{enumerate}
		\item распознавания символов;
		\item оценки устойчивости битумных эмульсий;
		\item автоматизации составления проектов организации дорожного движения (ПОДД). 
	\end{enumerate}
\end{enumerate}

\noindent
\underline{\textbf{Научная новизна:}}
\begin{enumerate}
\item Впервые построены математические модели контурных изображений, представленных ориентированными дугами, связям дуг и их численными характеристиками в градусном измерении,  а также  относительными размерами  длин дуг.  
\item Впервые  показано, что математические модели контурных изображений, хотя и имеют более сложную организацию данных, чем  таблицы реляционных БД, но позволяют получить почти аналогичные результаты по алгоритмической сложности  анализа контурных изображений, включая проверку изоморфных вложений образцов в анализируемое изображение.
\item Было выполнено оригинальное исследование использования масштабных рядов контурных изображений и процедуры сжатия на  основе относительных размеров дуг, которые можно использовать для повышения эффективности анализа исходных изображений.
\end{enumerate}
\noindent
\underline{\textbf{Практическая значимость}} диссертационной работы определяется
\begin{enumerate}
\item актуальностью направления исследований, которое обеспечивает применение информационных технологий буквально во всех сферах современной деятельности общества;
\item применением разработанных программных комплексов для решения прикладных задач, что подтверждено официальными справками о применении результатов диссертации.
Степень достоверности полученных результатов обеспечивается строгими математическими формулировками определений, а также строгими математическими доказательствами полученных утверждений, лемм и теорем.
\end{enumerate}

\noindent
\underline{\textbf{Достоверность}} Результаты находятся в соответствии с результатами, полученными другими авторами: А.И.Мальцевым, Ю.Л.Ершовым, С.В.Яблонским, А.И.Кокориным, А.В.Манциводой,  Коддом, Д.Кнутом, В.И.Мартьяновым, Д.В.Пахомовым, В.В.Архиповым и др.

\noindent
\underline{\textbf{Апробация работы.}}
Основные результаты работы докладывались на: 
\begin{enumerate}
	\item ежегодных научно-теоретических конференциях аспирантов и студентов:  Иркутский  гос. университет, ИМЭИ, 2010-13 гг.
	\item 3-ей Российской школе – семинаре «Синтаксис и семантика логических систем». Иркутск, 2010.
	\item 4-ой Международной конференции «Математика, ее приложения и математическое образование (МПМО’11),  Улан-Удэ, 2011.
	\item семинарах кафедр ИГУ, ИрГТУ, ВСГАО, 2010-14гг.
\end{enumerate}

%Диссертационная работа была выполнена при поддержке грантов ...
\noindent
\underline{\textbf{Личный вклад.}} Автором получены самостоятельно результаты основных положений \cite{A1,A2,A3,A4,A5,A6,A7,A8,A9},  выносимых  на защиту.  Результаты основных положений \cite{A3,A4,A5}  выносимых  на защиту, получены в нераздельном соавторстве с В.И. Мартьяновым, которому принадлежит начальное определение масштабных рядов контурных изображений и предложение их использования для уменьшения вычислительной сложности анализа контурных изображений.

\noindent
\underline{\textbf{Публикации.}} Основные результаты по теме диссертации изложены в 9 печатных изданиях \cite{A1,A2,A3,A4,A5,A6,A7,A8,A9}, 7 из которых изданы в журналах, рекомендованных ВАК \cite{A1,A2,A3,A4,A5,A6,A7,A8,A9}[1, 3, 5–9]. Диссертация состоит из введения, четырех глав, заключения и двух приложений. Полный объем диссертации составляет ХХХ страница с ХХ рисунками и ХХ таблицами. Список литературы содержит ХХХ наименований.

%\underline{\textbf{Объем и структура работы.}} Диссертация состоит из~введения, четырех глав, заключения и~приложения. Полный объем диссертации \textbf{ХХХ}~страниц текста с~\textbf{ХХ}~рисунками и~5~таблицами. Список литературы содержит \textbf{ХХX}~наименование.

%\newpage
\subsection*{\Large Содержание работы}
Во \underline{\textbf{введении}} обосновывается актуальность исследований, проводимых в рамках данной диссертационной работы, приводится обзор научной литературы по изучаемой проблеме, формулируется цель, ставятся задачи работы, сформулированы научная новизна и практическая значимость представляемой работы.

\underline{\textbf{Первая глава}} посвящена обзору существующих методов решения схожих задач. Таких как распознавание объектов на основе нейронных сетей. 

 картинку можно добавить так:
\begin{figure}[h] 
  \center
  \includegraphics [scale=0.27] {latex}
  \caption{Подпись к картинке.} 
  \label{img:latex}
\end{figure}

Формулы в строку без номера добавляются так:
$$
  \lambda_{T_s} = K_x\frac{d{x}}{d{T_s}}, \qquad
  \lambda_{q_s} = K_x\frac{d{x}}{d{q_s}},
$$

\underline{\textbf{Вторая глава}} посвящена исследованию 

\underline{\textbf{Третья глава}} посвящена исследованию 

В \underline{\textbf{четвертой главе}} приведено описание 

В \underline{\textbf{заключении}} приведены основные результаты работы, которые заключаются в следующем:
\begin{enumerate}
 \item Результат номер один.
 \item Результат номер два.
 \item Результат номер три.
% и так далее, если нужно
\end{enumerate}


%\newpage
\renewcommand{\refname}{\Large Публикации автора по теме диссертации}
\nocite{*}
\bibliography{biblio}
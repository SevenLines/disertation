%%% Макет страницы %%%
\oddsidemargin=-13pt
\topmargin=-66pt
\headheight=12pt
\headsep=38pt
\textheight=732pt
\textwidth=484pt
\marginparsep=14pt
\marginparwidth=43pt
\footskip=14pt
\marginparpush=7pt
\hoffset=0pt
\voffset=0pt
%\paperwidth=597pt
%\paperheight=845pt
\parindent=1.5cm %размер табуляции (для красной строки) в начале каждого абзаца
\renewcommand{\baselinestretch}{1.25}
\newfloat{scheme}{tb}{sch}

%%% Общая информация %%%
\author{Фамилия И.О.} % Фамилия И.О. автора

%%% Кодировки и шрифты %%%
\renewcommand{\rmdefault}{ftm} % Включаем Times New Roman

%%% Выравнивание и переносы %%%
\sloppy
\clubpenalty=10000
\widowpenalty=10000

%%% Библиография %%%
\makeatletter
\bibliographystyle{utf8gost705u} % Оформляем библиографию в соответствии с ГОСТ 7.0.5
\renewcommand{\@biblabel}[1]{#1.} % Заменяем библиографию с квадратных скобок на точку:
\makeatother

%%% Изображения %%%
\graphicspath{{images/}} % Пути к изображениям

%%% Цвета гиперссылок %%%
\definecolor{linkcolor}{rgb}{0,0,0}
\definecolor{citecolor}{rgb}{0,0,0}
\definecolor{urlcolor}{rgb}{0,0,1}
\hypersetup{
    colorlinks, linkcolor={linkcolor},
    citecolor={citecolor}, urlcolor={urlcolor}
}



%%% Математические высказывания %%%
\newtheorem{definition}{Определение} \numberwithin{definition}{section}
\newtheorem{theorem}{Теорема}
\newtheorem*{theorem*}{Теорема}
\newtheorem{lemma}{Лемма} \numberwithin{lemma}{section}
\newtheorem*{lemma*}{Лемма} 
\newtheorem{state}{Утверждение} \numberwithin{state}{section}
\newtheorem{proposition}{Предложение} \numberwithin{proposition}{section}
\newtheorem{corollary}{Следствие} \numberwithin{corollary}{section}

\setlist{nolistsep}
\setlist[itemize]{leftmargin=1.75cm}
\renewcommand\labelitemi{--}

\captionsetup[table]{labelsep=newline}
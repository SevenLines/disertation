\chapter*{ВВЕДЕНИЕ}							% Заголовок
\addcontentsline{toc}{chapter}{Введение}	% Добавляем его в оглавление

\textbf{Актуальность темы.}  Разработка эффективных вычислительных  методов решения информационных задач является одним из важнейших направлений в развитии современных информационных технологий.

Это можно отнести и к прогрессу в целом, сложно представить современную науку, технику  и общественную жизнь без использования электронных вычислительных устройств, подключенных к сети Интернет.

В идеале, такие методы обязаны обеспечивать скорость решения ряда важных информационных задач не зависящую от размеров данных. И действительно, есть ряд важных задач, где это достижимо. 

Одним из таких примеров (и очень важным) являются реляционные базы данных, где вычислимость запросов определенных типов не зависит от объема данных, а только линейно от сложности проекта самой БД.

Для простого обывателя практическим подтверждением этого является скорость работы банковских систем, с их мировыми сетями терминалов и банкоматов, которые позволяют осуществлять операции с вкладами и денежными средствами в любой точке земного шара за считанные секунды.

Стоит отметить, что достачно близко к классу таких задач примыкают задачи поиска данных по ключевым словам в сети Интернет (поисковые системы «Google», «Яндекс» и др.), где скорость поиска не замедляется из-за экспоненциального роста информации в глобальной сети.

Еще одним важным классом информационных задач являются вопросы распознавания образцов. Где при организации данных близкой к организации таблиц в реляционных базах данных также могут быть получены результаты доказывающие независимость скорости  распознавания  образцов от их количества, вернее, верхней границы сложности распознавания одного образца с добавкой только количества образцов (в частности, решению данных вопросов в значительной мере посвящена данная работа). 

Данное направление исследований (разработка эффективных вычислительных  методов решения информационных задач) весьма актуально для современной  науки в целом и техники, как в теоретическом, так и  практическом плане, где работают крупные транснациональные компьютерные корпорации, реализуются технологии «Big Table»,  «Big Data» и предполагается получение прорывных результатов в робототехнике, молекулярной биологии, системах искусственного интеллекта и других важных областях, имеющих определяющее значение для прогресса современного общества.

В диссертационной работе рассматриваются вопросы анализа контурных изображений (поиск образцов в изображении), получены оценки алгоритмической сложности такого анализа (близкие к аналогичным результатам для табличной организации данных), созданы программные комплексы решающие прикладные задачи.

Предложенные в  диссертационной работе методы и полученные результаты являются продолжением исследований: 
\begin{itemize}
\item Д.~Кнута и др. по эффективной вычислимости запросов для данных, представленных древовидными структурами;
\item Э.~Кодда и др. по табличной организации данных для реляционных сетевых СУБД (MS SQL Server, Oracle и др., включая технологии «Big Table»).
\end{itemize}

В принципиальном плане результаты диссертационной работы могут интерпретироваться как возможность получения для технологий «Big Data» по анализу изображений практически таких же эффективных методов, как и в   технологии  «Big Table» для табличной организации данных.


\textbf{Цель и задачи  исследования.} Целью  исследования является построение  моделей контурных изображений, которые, хотя и имеют более сложную организацию данных, чем  таблицы реляционных БД, но позволяют получить почти аналогичные результаты по вычислительной сложности  анализа контурных изображений, включая поиск изоморфных вложений образцов в анализируемое изображение.

\textit{Разработка новых  методов моделирования объектов и явлений}  в диссертации представлена результатами по построению  моделей контурных изображений, представленных двухосновными алгебраическими системами.

\textit{Разработка, обоснование и тестирование эффективных вычислительных методов с применением современных компьютерных технологий} в диссертации представлена результатами по эффективной вычислимости анализа, полученных  моделей контурных изображений, включая проверку изоморфной вложимости совокупности контурных изображений (образцы) в исследуемое контурное изображение.

\textit{Реализация эффективных численных методов и алгоритмов в виде комплексов проблемно-ориентированных программ для проведения вычислительного эксперимента} в диссертации представлена результатами по применению комплексов программ для решения конкретных прикладных задач  (см. Приложения).

\textbf{Для достижения поставленной цели необходимо было решить следующие задачи:}	

1. Разработать преобразование растра контурного изображения в нагруженный граф специального вида.

2. Разработать преобразование нагруженных графов специального вида в математические модели, представленные многоосновными алгебраическими системами, где контурные изображения сведены к ориентированным дугам, связям дуг и их численным характеристикам в градусном измерении и  относительных размеров  длины дуг.

3. Исследовать алгоритмическую сложность  анализа контурных изображений, включая проверку изоморфных вложений образцов в анализируемое изображение, где контурные изображения сведены к ориентированным дугам, связям дуг и их численным характеристикам в градусном измерении и  относительных размеров  длины дуг.

4. Разработать  масштабные ряды контурных изображений и процедуры сжатия на  основе относительных размеров дуг.

5. Исследовать возможность использования  изоморфного вложения сжатого образца в сжатое изображение для уменьшения алгоритмической сложности построения  изоморфного вложения исходного образца в исходное изображение.

6. Исследовать анализ плоских контурных изображений, представляющих объекты с наложениями.

7. Исследовать возможность использования полученных математических методов, математических моделей представления данных, алгоритмов и комплексов программ для решения прикладных задач:

\begin{itemize}
   \item распознавания рукописных символов;
   \item  оценки устойчивости битумных эмульсий;
   \item  автоматизации составления проектов организации дорожного движения (ПОДД). 
\end{itemize}

\textbf{Объект и предмет исследования.}  Объектом исследования являются плоские контурные изображения, представленные дугами, имеющими градусную меру и относительную длину. Предмет исследования – математические модели плоских контурных изображений, оценки сложности их анализа (поиск образцов в анализируемом изображении), программная реализация анализа изображений и применение для решения прикладных задач. 
	
\textbf{Методология и методы исследования.} Для задания основных свойств  плоских контурных изображений и их построения из растровых изображений использовался  принцип математического моделирования. Исследование принятых моделей, представленных двухосновными алгебраическими системами, выполнялось на основе численных методов построения нагруженных графов, представленных дугами, имеющими градусную меру и относительную длину, а также связями дуг, имеющими градусную меру. Разработка авторских комплексов программ проводилась в среде Borland C++ Builder. 


\textbf{Достоверность  результатов.} Степень достоверности полученных результатов обеспечивается строгими математическими формулировками определений, а также строгими математическими доказательствами полученных утверждений, лемм и теорем.
Результаты находятся в соответствии с результатами, полученными другими авторами:  А.И.~Мальцевым\cite{D12}, Ю.Л.~Ершовым\cite{D12}, С.В.~Яблонским\cite{D21}, А.И.~Кокориным\cite{D10}, А.В.~Манциводой\cite{D13},  Коддом\cite{D25}, Д.~Кнутом\cite{D9}, В.И.~Мартьяновым\cite{Samara}, Д.В.~Пахомовым\cite{D16}, В.В.~Архиповым\cite{D15} и др\cite{D1,D2,D3,D11}.

\textbf{Научная новизна} работы заключается в следующем: 

1. Построены математические модели контурных изображений, представленных ориентированными дугами, связям дуг и их численными характеристиками в градусном измерении,  а также  относительными размерами  длин дуг.  

2. Показано, что математические модели контурных изображений, хотя и имеют более сложную организацию данных, чем  таблицы реляционных БД, но позволяют получить почти аналогичные результаты по алгоритмической сложности  анализа контурных изображений, включая проверку изоморфных вложений образцов в анализируемое изображение.

3. Проведены исследования по использованию масштабных рядов контурных изображений и процедуры сжатия на  основе относительных размеров дуг, которые можно использовать для повышения эффективности анализа исходных изображений.

4. Исследован анализ плоских контурных изображений, представляющих объекты с наложениями.

\textbf{Практическая значимость работы.}

1. Разработан  программный комплекс распознавания рукописных символов.

2. Разработан  программный комплекс оценки устойчивости битумных эмульсий, внедренный в технопарке  ИРНИТУ (см. Приложение Б \ref{AppendixB}).

3. Разработан  программный комплекс автоматизации составления проектов организации дорожного движения (ПОДД), внедренный в ОГКУ «Дирекция автомобильных дорог» Администрации Иркутской области (см. Приложение Б \ref{AppendixB}).

\textbf{Апробация работы.} Основные результаты работы докладывались на: 


1. Ежегодных научно-теоретических конференциях аспирантов и студентов:  Иркутский  гос. университет, ИМЭИ, 2010--13 гг.

2. 3-ей Российской школе – семинаре «Синтаксис и семантика логических систем». Иркутск, 2010.

3. 4-ой Международной конференции «Математика, ее приложения и математическое образование (МПМО’11),  Улан-Удэ, 2011.

4. Семинарах кафедр ИГУ, ИРНИТУ, ИрГУПС, 2010--17 гг.

5. Межрегиональных конференциях "Платоновские чтения --- 2015--17 гг.


\textbf{Личный вклад.} Автором получены самостоятельно результаты основных положений 1, 2, 6,  выносимых  на защиту.  Результаты основных положений 3, 4, 5,  выносимых  на защиту, получены в нераздельном соавторстве с В.И. Мартьяновым, которому принадлежит начальное определение масштабных рядов контурных изображений и предложение их использования для уменьшения вычислительной сложности анализа контурных изображений.


\textbf{Публикации.} Результаты диссертационного исследования
опубликованы в 11 научных работах \cite{D5,D6,D7,D8,D15,D16,D19,D20,scaleline,overlaps,emulsion}, в том числе 8 \cite{D6,D7,D8,D15,D19,D20,scaleline,overlaps,emulsion} работ в рецензируемых научных изданиях, рекомендованных ВАК РФ. Получено свидетельство о государственной регистрации программ для ЭВМ \cite{D18}.
Все вносимые на защиту результаты получены лично автором или при его непосредственном участии.


Объем и структура работы. Диссертация состоит из введения, четырех глав, заключения и двух приложений. Полный объем диссертации составляет 100 страниц с \totalfigures\ рисунками и \totaltables\ таблицами. Список литературы содержит 38 наименований.
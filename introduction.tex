\chapter*{Введение}							% Заголовок
\addcontentsline{toc}{chapter}{Введение}	% Добавляем его в оглавление
%Сегодня существует огромное количество систем в той или иной степени успешно решающих задачу распознования изображений. Это может быть распознование лиц, отпечатков пальцов, топографических планов, рентгеновских снимков. Но в наиболее промышленных масштабах данная технология используется для распознования текста. Бланки ЕГЭ, ГИА, Банки -- это лишь небольшая часть областей где используют распознование текста.
%Рассмотрим некоторые наиболее популярные системы распознования текста:
%
%Все данные системы 

%Обзор, введение в тему, обозначение места данной работы в мировых исследованиях и т.п.

Важнейшим направлением развития современных информационных технологий (и в целом, прогресса, так как невозможно представить современную науку и технику  без использования компьютеров) является создание математических методов быстрого решения информационных задач. В идеале, эти методы должны обеспечивать скорость решения ряда важных информационных задач вне зависимости от объема данных. И действительно, есть ряд важных  информационных задач, где это возможно. 

Наглядным (и очень важным) примером  этого являются реляционные базы данных (БД), где вычислимость запросов определенных типов не зависит от объема данных, а только линейно от сложности проекта самой БД.Практическим подтверждением этого для обывателя является скорость работы банковских систем, использующих сетевые реляционные БД,  с их мировыми сетями терминалов и банкоматов, где можно проводить операции с вкладами и денежными средствами в любой точке мира за считанные секунды.

Отметим, что весьма близко к этому классу  примыкают задачи поиска данных по ключевым словам в Интернет - пространстве (полно текстовые поисковые системы <<Google>>, <<Яндекс>> и др.), где скорость поиска не замедляется из-за экспоненциального роста информации в глобальной сети.


Другим  важным классом информационных задач являются вопросы распознавания  образцов (образов), где при организации данных близкой к таблицам реляционных БД также могут быть получены результаты независимости скорости  распознавания  образцов от их количества, вернее, верхней границы сложности распознавания одного образца с добавкой только количества образцов. 

Данное направление исследований весьма актуально для современной математики, науки в целом и техники, как в теоретическом, так и  практическом плане, где работают крупные транснациональные корпорации, реализуются технологии <<Big Table>>, <<Big Data>> и предполагается получение прорывных результатов в робототехнике, молекулярной биологии, системах искусственного интеллекта и других важных областях для прогресса современного общества.

Целью данной работы является построение математических моделей контурных изображений, которые, хотя и имеют более сложную организацию данных, чем  таблицы реляционных БД, но позволяют получить почти аналогичные результаты по алгоритмической сложности  анализа контурных изображений, включая проверку изоморфных вложений образцов в анализируемое изображение.
Для достижения поставленной цели необходимо было решить следующие задачи:	
\begin{enumerate}
	\item Разработать преобразование растра контурного изображения в нагруженный граф специального вида.
	\item Разработать преобразование нагруженных графов специального вида в математические модели, представленные многоосновными алгебраическими системами, где контурные изображения сведены к ориентированным дугам, связям дуг и их численным характеристикам в градусном измерении и  относительных размеров  длины дуг.
	\item Исследовать алгоритмическую сложность  анализа контурных изображений, включая проверку изоморфных вложений образцов в анализируемое изображение, где контурные изображения сведены к ориентированным дугам, связям дуг и их численным характеристикам в градусном измерении и  относительных размеров  длины дуг.
	\item Разработать  масштабные ряды контурных изображений и процедуры сжатия на  основе относительных размеров дуг.
	\item Исследовать возможность использования  изоморфного вложения сжатого образца в сжатое изображение для уменьшения алгоритмической сложности построения  изоморфного вложения исходного образца в исходное изображение.
	\item Исследовать возможность использования полученных математических методов, математических моделей представления данных, алгоритмов и комплексов программ для решения прикладных задач: 
	\begin{enumerate}
		\item распознавания символов
		\item оценки устойчивости битумных эмульсий
%		\item автоматизации составления проектов организации дорожного движения (ПОДД)
	\end{enumerate}
\end{enumerate}

\subsection*{Основные положения, выносимые на защиту}
\begin{enumerate}
	\item Сходящийся алгоритм волновой скелетизации,  обеспечивающий преобразование растра контурного изображения в нагруженный граф специального вида (Утверждение 2.3.1.).
	\item Оценка нижней границы алгоритмической сложности  анализа контурных изображений, включая проверку изоморфных вложений образцов в анализируемое изображение, где контурные изображения сведены к ориентированным дугам, связям дуг и их численным характеристикам в градусном измерении и  относительным размерам  длины дуг  (Теорема 2.5.1.).
	\item Необходимое и достаточное условие продолжения изоморфного вложения сжатого образца в сжатое изображение до изоморфного вложения исходного образца в исходное изображение  (Теорема 1.).
	\item Верхняя граница алгоритмической сложности построения продолжения изоморфного вложения сжатого образца в сжатое изображение до изоморфного вложения исходного образца в исходное изображение  (Теорема 2.).
	\item Уменьшение верхней границы алгоритмической сложности построения  изоморфного вложения исходного образца в исходное изображение при использовании изоморфного вложения сжатого образца в сжатое изображение  (Теорема 3.).
	\item Комплексы программ  решения прикладных задач:
	\begin{enumerate}
		\item распознавания символов
		\item оценки устойчивости битумных эмульсий
	\end{enumerate}
\end{enumerate}

\subsection*{Научная новизна}
\begin{enumerate}
	\item Впервые построены математические модели контурных изображений, представленных ориентированными дугами, связям дуг и их численными характеристиками в градусном измерении,  а также  относительными размерами  длин дуг.  
	\item Впервые  показано, что математические модели контурных изображений, хотя и имеют более сложную организацию данных, чем  таблицы реляционных БД, но позволяют получить почти аналогичные результаты по алгоритмической сложности  анализа контурных изображений, включая проверку изоморфных вложений образцов в анализируемое изображение.
	\item Было выполнено оригинальное исследование использования масштабных рядов контурных изображений и процедуры сжатия на  основе относительных размеров дуг, которые можно использовать для повышения эффективности анализа исходных изображений.
\end{enumerate}

\subsection*{Научная и практическая значимость}

определяется во-первых, актуальностью направления исследований, которые обеспечивают применение информационных технологий буквально во всех сферах современной деятельности общества;

Во-вторых, применением разработанных программных комплексов для решения прикладных задач, что подтверждено официальными справками о применении результатов диссертации.
Степень достоверности полученных результатов обеспечивается строгими математическими формулировками определений, а также строгими математическими доказательствами полученных утверждений, лемм и теорем.

Результаты находятся в соответствии с результатами, полученными другими авторами: А.И.Мальцевым, Ю.Л.Ершовым, С.В.Яблонским, А.И.Кокориным, А.В.Манциводой,  Коддом, Д.Кнутом, В.И.Мартьяновым, Д.В.Пахомовым, В.В.Архиповым и др.

%
%\textbf{Целью} данной работы является \ldots
%
%Для~достижения поставленной цели необходимо было решить следующие задачи:
%\begin{enumerate}
%  \item Исследовать, разработать, вычислить и т.д. и т.п.
%  \item Исследовать, разработать, вычислить и т.д. и т.п.
%  \item Исследовать, разработать, вычислить и т.д. и т.п.
%  \item Исследовать, разработать, вычислить и т.д. и т.п.
%\end{enumerate}
%
%\textbf{Основные положения, выносимые на~защиту:}
%\begin{enumerate}
%  \item Первое положение
%  \item Второе положение
%  \item Третье положение
%  \item Четвертое положение
%\end{enumerate}
%
%\textbf{Научная новизна:}
%\begin{enumerate}
%  \item Впервые \ldots
%  \item Впервые \ldots
%  \item Было выполнено оригинальное исследование \ldots
%\end{enumerate}
%
%\textbf{Научная и практическая значимость} \ldots
%
%\textbf{Степень достоверности} полученных результатов обеспечивается \ldots Результаты находятся в соответствии с результатами, полученными другими авторами.
%
%\textbf{Апробация работы.}
%Основные результаты работы докладывались~на:
%перечисление основных конференций, симпозиумов и т.п.
%
%\textbf{Личный вклад.} Автор принимал активное участие \ldots
%
%\textbf{Публикации.} Основные результаты по теме диссертации изложены в ХХ печатных изданиях~\cite{Main1,Main2,WaveSkeletization},
%Х из которых изданы в журналах, рекомендованных ВАК~\cite{Main1,Main2,WaveSkeletization}, 
%%ХХ --- в тезисах докладов~\cite{bib4,bib5}.

\textbf{Объем и структура работы.} Диссертация состоит из~введения, трех глав, заключения и~двух приложений. Полный объем диссертации составляет ХХХ~страница с~ХХ~рисунками и~ХХ~таблицами. Список литературы содержит ХХХ~наименований.

\clearpage
\chapter*{Введение}							% Заголовок
\addcontentsline{toc}{chapter}{Введение}	% Добавляем его в оглавление

Разработка эффективных  математических методов решения информационных задач является важнейшим направлением развития современной  дискретной математики и математической кибернетики (и в целом, прогресса, так как невозможно представить современную науку, технику, да и частную жизнь человека  без использования компьютеров). 
В идеале, эти методы должны обеспечивать скорость решения  информационных задач вне зависимости от объема данных. И действительно, есть ряд важных  информационных задач, где это возможно. 

Наглядным (и очень важным) примером  этого являются реляционные базы данных (БД), где вычислимость запросов определенных типов не зависит от объема данных, а только линейно от сложности проекта самой БД.

Практическим подтверждением этого для любого человека является скорость работы банковских систем, использующих сетевые реляционные БД,  с их мировыми сетями терминалов и банкоматов, где можно проводить операции с вкладами и денежными средствами в любой точке мира за считанные секунды.

Отметим, что весьма близко к этому классу  примыкают задачи поиска данных по ключевым словам в Интернет - пространстве (полно текстовые поисковые системы «Google» \cite{BigData}, «Яндекс» и др.), где скорость поиска не замедляется из-за экспоненциального роста информации в глобальной сети.

Другим  важным классом информационных задач являются вопросы распознавания  образцов (образов), где при организации данных близкой к таблицам реляционных БД также могут быть получены результаты независимости скорости  распознавания  образцов от их количества \cite{Samara}, вернее, верхней границы сложности распознавания одного образца с добавкой только количества образцов. 

Данное направление исследований (разработка эффективных вычислительных математических методов решения информационных задач) весьма актуально для современной дискретной математики, математической кибернетики, науки в целом и техники, как в теоретическом, так и  практическом плане, где работают крупные транснациональные компьютерные корпорации, реализуются технологии «Big Table»,  «Big Data» и предполагается получение прорывных результатов в робототехнике, молекулярной биологии, системах искусственного интеллекта и других важных областях, имеющих определяющее значение для прогресса современного общества \cite{BigTable,BigData}.
Целью данной работы является построение конечных дискретных моделей контурных изображений, которые, хотя и имеют более сложную организацию, чем  таблицы реляционных БД, но позволяют получить почти аналогичные результаты по  сложности  алгоритмов анализа контурных изображений, включая проверку изоморфных вложений образцов (распознавание) в анализируемое изображение.

Разработка новых  методов дискретной математики и  кибернетики для моделирования объектов и явлений  в диссертации представлена результатами по построению конечных дискретных моделей контурных изображений.

Разработка, обоснование и тестирование эффективных вычислительных методов дискретной математики и  кибернетики с применением современных компьютерных технологий в диссертации представлена результатами по эффективной вычислимости распознавания, полученных конечных дискретных  моделей контурных изображений, включая проверку изоморфной вложимости совокупности контурных изображений (образцы) в исследуемое контурное изображение.

Реализация эффективных численных методов и алгоритмов дискретной математики и  кибернетики в виде комплексов проблемно-ориентированных программ для проведения вычислительного эксперимента в диссертации представлена результатами по применению комплексов программ для решения конкретных прикладных задач  (см. Приложения).

\noindent
\textbf{Для достижения поставленной цели необходимо было решить следующие задач: }
\begin{enumerate}
\item Разработать преобразование растра контурного изображения в нагруженный граф специального вида.
\item Разработать преобразование нагруженных графов специального вида в математические модели, представленные многоосновными алгебраическими системами, где контурные изображения сведены к ориентированным дугам, связям дуг и их численным характеристикам в градусном измерении и  относительных размеров  длины дуг.
\item Исследовать алгоритмическую сложность  анализа контурных изображений, включая проверку изоморфных вложений образцов в анализируемое изображение, где контурные изображения сведены к ориентированным дугам, связям дуг и их численным характеристикам в градусном измерении и  относительных размеров  длины дуг.
\item Разработать  масштабные ряды контурных изображений и процедуры сжатия на  основе относительных размеров дуг.
\item Исследовать возможность использования  изоморфного вложения сжатого образца в сжатое изображение для уменьшения алгоритмической сложности построения  изоморфного вложения исходного образца в исходное изображение.
\item Исследовать алгоритмическую сложность  анализа плоских контурных изображений, представляющих объекты с наложениями. Т.е. предполагается наличие объектов первого плана, отображенных на изображении без каких-либо искажений,  а также объектов второго и других планов, которые в той или иной степени закрыты более близко стоящими к точке съемки объектами.
\item Исследовать возможность использования полученных математических методов, математических моделей представления данных, алгоритмов и комплексов программ для решения прикладных задач:
\begin{enumerate}
\item распознавания символов;
\item  оценки устойчивости битумных эмульсий; 
\item автоматизации составления проектов организации дорожного движения (ПОДД). 
\end{enumerate}

\end{enumerate}
 

\textbf{Основные положения, выносимые на защиту:}
\begin{enumerate}

\item Сходящийся алгоритм волновой скелетизации,  обеспечивающий преобразование растра контурного изображения в нагруженный граф специального вида (Утверждение \ref{WaveSkeletizationState}).
\item Оценка нижней границы алгоритмической сложности  анализа контурных изображений, включая проверку изоморфных вложений образцов в анализируемое изображение, где контурные изображения сведены к ориентированным дугам, связям дуг и их численным характеристикам в градусном измерении и  относительным размерам  длины дуг  (\ref{main_theorem_1}).
\item Масштабные ряды плоских контурных изображений и их применение для уменьшения  вычислительной сложности анализа контурных изображений.  (Теоремы \ref{scale:theoremA}, \ref{scale:theoremB}, \ref{scale:theorem:3}).
\item Оценка нижней границы алгоритмической сложности  анализа  плоских контурных изображений, представляющих объекты с наложениями. Т.е. предполагается наличие объектов первого плана, отображенных на изображении без каких-либо искажений,  а также объектов второго и других планов, которые в той или иной степени закрыты более близко стоящими к точке съемки объектами (Теоремы \ref{overlaps:t:5})
\item Комплексы программ  решения прикладных задач:
	\begin{enumerate}
	\item распознавания символов;
	\item оценки устойчивости битумных эмульсий;
	\item автоматизации составления проектов организации дорожного движения (ПОДД). 
	\end{enumerate}   
\end{enumerate}
   
\textbf{Научная новизна:}
\begin{enumerate}
\item Впервые построены математические модели контурных изображений, представленных ориентированными дугами, связям дуг и их численными характеристиками в градусном измерении,  а также  относительными размерами  длин дуг.  
\item Впервые  показано, что конечные дискретные модели контурных изображений, хотя и имеют более сложную организацию данных, чем  таблицы реляционных БД, но позволяют получить почти аналогичные результаты по алгоритмической сложности  анализа контурных изображений, включая проверку изоморфных вложений образцов в анализируемое изображение.
\item Было выполнено оригинальное исследование использования масштабных рядов контурных изображений и процедуры сжатия на  основе относительных размеров дуг, которые можно использовать для повышения эффективности анализа исходных изображений.
\item Было выполнено оригинальное исследование анализа плоских контурных изображений, представляющих объекты с наложениями. Т.е. предполагается наличие объектов первого плана, отображенных на изображении без каких-либо искажений,  а также объектов второго и других планов, которые в той или иной степени закрыты более близко стоящими к точке съемки объектами.
\end{enumerate}

\noindent
\textbf{Научная и практическая значимость определяется:}
\begin{itemize}
\item Во-первых, актуальностью направления исследований, которое обеспечивает применение информационных технологий буквально во всех сферах деятельности современного общества;
\item Во-вторых, применением разработанных программных комплексов для решения прикладных задач, что подтверждено официальными справками о внедрении результатов диссертации.
\end{itemize}

Степень достоверности полученных результатов обеспечивается строгими математическими формулировками определений, а также строгими математическими доказательствами полученных утверждений, лемм и теорем.
Результаты находятся в соответствии с результатами, полученными другими авторами:  А.И.Мальцевым\cite{D12}, Ю.Л.Ершовым\cite{D12}, С.В.Яблонским\cite{D21}, А.И.Кокориным\cite{D10}, А.В.Манциводой\cite{D13},  Коддом\cite{D25}, Д.Кнутом\cite{D9}, В.И.Мартьяновым\cite{Samara}, Д.В.Пахомовым\cite{D16}, В.В.Архиповым\cite{D15} и др\cite{D1,D2,D3,D11}.

\noindent
\textbf{Апробация работы.} Основные результаты работы докладывались на: 
\begin{enumerate}
\item ежегодных научно-теоретических конференциях аспирантов и студентов:  Иркутский  гос. университет, ИМЭИ, 2010-13 гг.
\item 3-ей Российской школе – семинаре «Синтаксис и семантика логических систем». Иркутск, 2010.
\item 4-ой Международной конференции «Математика, ее приложения и математическое   
    образование (МПМО’11),  Улан-Удэ, 2011.
\item Межрегиональных конференциях "Платоновские чтения – 2015 и 2016".
\item Семинарах кафедр ИГУ, ИРНИТУ, ВСГАО, 2010-16гг.
\end{enumerate}

\textbf{Личный вклад.} Автором получены самостоятельно результаты основных положений 1, 2, 7,  выносимых  на защиту.  Результаты основных положений 3, 4, 5, 6  выносимых  на защиту, получены в нераздельном соавторстве с В.И. Мартьяновым, которому принадлежит начальное определение масштабных рядов контурных изображений и предложение их использования для уменьшения вычислительной сложности анализа контурных изображений.

Публикации. Основные результаты по теме диссертации изложены в 10 печатных
изданиях \cite{D5,D6,D7,D8, D15,D16,D18,D19,D20,scaleline}, 7 из которых изданы в журналах, рекомендованных ВАК \cite{D6,D7,D8, D15,D19,D20,scaleline}.
Объем и структура работы. Диссертация состоит из введения, четырех глав, заключения и двух приложений. Полный объем диссертации составляет \pageref{LastPage}\ страниц с \totalfigures\ рисунками и \totaltables\ таблицами. Список литературы содержит ХХХ наименований.

%Сегодня существует огромное количество систем в той или иной степени успешно решающих задачу распознования изображений. Это может быть распознование лиц, отпечатков пальцов, топографических планов, рентгеновских снимков. Но в наиболее промышленных масштабах данная технология используется для распознования текста. Бланки ЕГЭ, ГИА, Банки -- это лишь небольшая часть областей где используют распознование текста.
%Рассмотрим некоторые наиболее популярные системы распознования текста:
%
%Все данные системы 

%Обзор, введение в тему, обозначение места данной работы в мировых исследованиях и т.п.

%Разработка эффективных вычислительных математических методов решения информационных задач является важнейшим направлением развития современных информационных технологий (и в целом, прогресса, так как невозможно представить современную науку и технику  без использования компьютеров). 
%
%В идеале, эти методы должны обеспечивать скорость решения ряда важных информационных задач вне зависимости от объема данных. И действительно, есть ряд важных  информационных задач, где это возможно. 
%
%Наглядным (и очень важным) примером  этого являются реляционные базы данных (БД), где вычислимость запросов определенных типов не зависит от объема данных, а только линейно от сложности проекта самой БД.
%
%Практическим подтверждением этого для обывателя является скорость работы банковских систем, использующих сетевые реляционные БД,  с их мировыми сетями терминалов и банкоматов, где можно проводить операции с вкладами и денежными средствами в любой точке мира за считанные секунды.
%
%Отметим, что весьма близко к этому классу  примыкают задачи поиска данных по ключевым словам в Интернет - пространстве (полно текстовые поисковые системы «Google»\cite{D23}, «Яндекс» и др.), где скорость поиска не замедляется из-за экспоненциального роста информации в глобальной сети.
%
%Другим  важным классом информационных задач являются вопросы распознавания  образцов (образов), где при организации данных близкой к таблицам реляционных БД также могут быть получены результаты независимости скорости  распознавания  образцов от их количества\cite{D14}, вернее, верхней границы сложности распознавания одного образца с добавкой только количества образцов. 
%
%Данное направление исследований (разработка эффективных вычислительных математических методов решения информационных задач) весьма актуально для современной математики, науки в целом и техники, как в теоретическом, так и  практическом плане, где работают крупные транснациональные компьютерные корпорации, реализуются технологии «Big Table»,  «Big Data» и предполагается получение прорывных результатов в робототехнике, молекулярной биологии, системах искусственного интеллекта и других важных областях, имеющих определяющее значение для прогресса современного общества\cite{D22, D23}.
%
%\textbf{Целью данной работы является} построение математических моделей контурных изображений, которые, хотя и имеют более сложную организацию данных, чем  таблицы реляционных БД, но позволяют получить почти аналогичные результаты по вычислительной сложности  анализа контурных изображений, включая проверку изоморфных вложений образцов в анализируемое изображение.
%
%Разработка новых математических методов моделирования объектов и явлений, в диссертации представлена результатами по построению математических моделей контурных изображений.
%
%Разработка, обоснование и тестирование эффективных вычислительных методов с применением современных компьютерных технологий в диссертации представлена результатами по эффективной вычислимости анализа, полученных математических моделей контурных изображений, включая проверку изоморфной вложимости совокупности контурных изображений (образцы) в исследуемое контурное изображение.
%
%Реализация эффективных численных методов и алгоритмов в виде комплексов проблемно-ориентированных программ для проведения вычислительного эксперимента в диссертации представлена результатами по применению комплексов программ для решения конкретных прикладных задач.
%
%\noindent
%Для достижения поставленной цели необходимо было решить следующие задачи:	
%\begin{enumerate}
%	\item Разработать преобразование растра контурного изображения в нагруженный граф специального вида.
%	\item Разработать преобразование нагруженных графов специального вида в математические модели, представленные многоосновными алгебраическими системами, где контурные изображения сведены к ориентированным дугам, связям дуг и их численным характеристикам в градусном измерении и  относительных размеров  длины дуг.
%	\item Исследовать алгоритмическую сложность  анализа контурных изображений, включая проверку изоморфных вложений образцов в анализируемое изображение, где контурные изображения сведены к ориентированным дугам, связям дуг и их численным характеристикам в градусном измерении и  относительных размеров  длины дуг.
%	\item Разработать  масштабные ряды контурных изображений и процедуры сжатия на  основе относительных размеров дуг.
%	\item Исследовать возможность использования  изоморфного вложения сжатого образца в сжатое изображение для уменьшения алгоритмической сложности построения  изоморфного вложения исходного образца в исходное изображение.
%	\item Исследовать возможность использования полученных математических методов, математических моделей представления данных, алгоритмов и комплексов программ для решения прикладных задач:
%	\begin{enumerate}
%		\item распознавания символов;
%		\item оценки устойчивости битумных эмульсий;
%		\item автоматизации составления проектов организации дорожного движения (ПОДД). 
%	\end{enumerate}
%\end{enumerate}
%\noindent
%Основные положения, выносимые на защиту:
%\begin{enumerate}
%	\item Сходящийся алгоритм волновой скелетизации,  обеспечивающий преобразование растра контурного изображения в нагруженный граф специального вида (Утверждение 2.3.1.).
%	\item Оценка нижней границы алгоритмической сложности  анализа контурных изображений, включая проверку изоморфных вложений образцов в анализируемое изображение, где контурные изображения сведены к ориентированным дугам, связям дуг и их численным характеристикам в градусном измерении и  относительным размерам  длины дуг  (Теорема 2.5.1.).
%%	\item Необходимое и достаточное условие продолжения изоморфного вложения сжатого образца в сжатое изображение до изоморфного вложения исходного образца в исходное изображение  (Теорема 1.).
%%	\item Верхняя граница алгоритмической сложности построения продолжения изоморфного вложения сжатого образца в сжатое изображение до изоморфного вложения исходного образца в исходное изображение  (Теорема 2.).
%%	\item Уменьшение верхней границы алгоритмической сложности построения  изоморфного вложения исходного образца в исходное изображение при использовании изоморфного вложения сжатого образца в сжатое изображение  (Теорема 3.).
%	\item Комплексы программ  решения прикладных задач:
%	\begin{enumerate}
%		\item распознавания символов;
%		\item оценки устойчивости битумных эмульсий;
%		\item автоматизации составления проектов организации дорожного движения (ПОДД). 
%	\end{enumerate}
%\end{enumerate}
%\noindent
%Научная новизна:
%\begin{enumerate}
%\item Впервые построены математические модели контурных изображений, представленных ориентированными дугами, связям дуг и их численными характеристиками в градусном измерении,  а также  относительными размерами  длин дуг.  
%\item Впервые  показано, что математические модели контурных изображений, хотя и имеют более сложную организацию данных, чем  таблицы реляционных БД, но позволяют получить почти аналогичные результаты по алгоритмической сложности  анализа контурных изображений, включая проверку изоморфных вложений образцов в анализируемое изображение.
%%\item Было выполнено оригинальное исследование использования масштабных рядов контурных изображений и процедуры сжатия на  основе относительных размеров дуг, которые можно использовать для повышения эффективности анализа исходных изображений.
%\end{enumerate}
%\noindent
%Научная и практическая значимость определяется:
%\begin{enumerate}
%\item актуальностью направления исследований, которое обеспечивает применение информационных технологий буквально во всех сферах современной деятельности общества;
%\item применением разработанных программных комплексов для решения прикладных задач, что подтверждено официальными справками о применении результатов диссертации.
%Степень достоверности полученных результатов обеспечивается строгими математическими формулировками определений, а также строгими математическими доказательствами полученных утверждений, лемм и теорем.
%\end{enumerate}
%Результаты находятся в соответствии с результатами, полученными другими авторами: А.И.Мальцевым\cite{D12}, Ю.Л.Ершовым\cite{D12}, С.В.Яблонским\cite{D21}, А.И.Кокориным\cite{D10}, А.В.Манциводой\cite{D13},  Коддом\cite{D24,D25}, Д.Кнутом\cite{D9}, В.И.Мартьяновым\cite{D14}, Д.В.Пахомовым\cite{D16}, В.В.Архиповым\cite{D15} и др\cite{D1,D2,D3,D11}.
%
%
%\noindent
%Основные результаты работы докладывались на: 
%\begin{enumerate}
%	\item ежегодных научно-теоретических конференциях аспирантов и студентов:  Иркутский  гос. университет, ИМЭИ, 2010-13 гг;
%	\item 3-ей Российской школе – семинаре «Синтаксис и семантика логических систем». Иркутск, 2010;
%	\item 4-ой Международной конференции «Математика, ее приложения и математическое образование (МПМО’11),  Улан-Удэ, 2011;
%	\item семинарах кафедр ИГУ, ИрГТУ, ВСГАО, 2010-14гг;
%\end{enumerate}
%
%Автором получены самостоятельно результаты основных положений 1, 2,  выносимых  на защиту.  Результаты основных положений 3,4  выносимых  на защиту, получены в нераздельном соавторстве с В.И. Мартьяновым, которому принадлежит начальное определение масштабных рядов контурных изображений и предложение их использования для уменьшения вычислительной сложности анализа контурных изображений.
%
%Основные результаты по теме диссертации изложены в 10 печатных изданиях \cite{D5,D6,D7,D8, D15,D16,D17,D18,D19,D20}, 7 из которых изданы в журналах, рекомендованных ВАК \cite{D6,D7,D8, D15,D17,D19,D20}. Диссертация состоит из введения, четырех глав, заключения и двух приложений. Полный объем диссертации составляет \pageref{LastPage}\ страниц с \totalfigures\ рисунками и \totaltables\ таблицами. Список литературы содержит ХХХ наименований.
%\clearpage
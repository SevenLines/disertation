\chapter{Распознавание изображений}
\section{Базовые понятия}
\begin{definition}
Растровое изображение есть функция $$I_{rgb}(x,y): N \times N \to([0,255],[0,255],[0,255])$$ 
\end{definition}
Таким образом каждой точке $(x,y)$ мы сопоставляем тройку $(r,g,b)$. Первый элемент тройки соответствует красной компоненте цвета в растровом изображении, второй -- зеленой и третий -- синей. Далее, для краткости, будем использовать следующую запись:

$$
\begin{array}{l}
I_r(x,y) = I_{rgb}(x,y)_r \text{ -- красная компонента}\\
I_g(x,y) = I_{rgb}(x,y)_g \text{ -- зеленая компонента}\\
I_b(x,y) = I_{rgb}(x,y)_b \text{ -- синяя компонента}\\
\end{array}
$$

\begin{definition}
Определим растровое изображение заданного в оттенках серого как функцию
$$I_{grey}(x,y): N \times N \to[0,255]$$
\end{definition}

\begin{definition}
Введем оператор <<обесцвечивания>> $D$, который обеспечивает переход от цветного изображения $I_{rgb}$ к изображению $I_{grey}$, заданному в оттенках серого:
$$D(I_{rgb}) = I_{grey}$$
\end{definition}

Существует несколько основных способов обесцвечивания изображения:
\begin{enumerate}
 \item Красный канал $D_{red}(I_{rgb}) = I_r$
 \item Зеленый канал $D_{greeb}(I_{rgb}) = I_g$
 \item Синий канал $D_{blue}(I_{rgb}) = I_b$  
 \item Среднее значение (average): $D_{avg}(I_{rgb}) = \frac{I_r + I_g + I_b}{3}$
 \item Лума (luma), учитывает особенности восприятия цвета человеком: $$D_{luma}(I_rgb) = I_r\cdot0.3 + I_g\cdot0.59 + I_b\cdot0.11$$
 		Значения коэффициентов, иногда, могут отличаться от приведенных выше, но их сумма всегда равна 1
 \item Минимум  $D_{min}(I_{rgb}) = min(I_r, I_g, I_b)$
 \item Максимум  $D_{min}(I_{rgb}) = max(I_r, I_g, I_b)$
 \item Обесцвечивание (desaturtaion): $D_{desaturation}(I_{rgb}) = \frac{D_{min}(I_{rgb}) + D_{max}(I_{rgb})}{2}$
\end{enumerate}

%TODO вставить картинку сравнения методов обесцвечивания

\begin{remark}
Наилучший результат для средне-статистического изображения (с гистограммой близкой к нормальной) получается при использование 5-го и последнего способов. Под наилучшим результатом понимается сохранение яркости (компоненты value в модели HSV) цветов исходного изображения.
\end{remark}

\begin{definition}
Растровое монохромное изображение есть функция 
$$I_m(x,y): N \times N \to \{0,1\}$$
\end{definition}

Переход от изображения заданного в оттенках серого к монохромному изображению осуществляется через операцию отсечения. Операция отсечения реализуется через оператор отсечения $T$, для некоторого фиксированного $t\in[0,255]$
\begin{definition}
Оператор отсечения $T$ есть:
$$
T(t, I_{grey}) = \left\{ 
\begin{array}{ll}
1 & ,I_{grey} < t \\
0 & ,\text{иначе} 
\end{array}
\right\}
$$
\end{definition}
Таким образом монохромное изображение есть $I_m = T(t, I_{grey})$

\begin{remark}
Значение $t$ определяется опытном путем и зависит от исходного изображения. Для рукописного текста написанного черной ручкой на офисной бумаге берутся значения близкие к 100 (чем меньше значение, тем темнее изображение).
\end{remark}
%TODO последовательность изображений показывающих переход от цветного изображения -> к серому -> к монохромному

\begin{remark}
В данной работе не рассматриваются методы адаптивного отсечения, в силу их медленной производительности излишней в данном контексте точности.
\end{remark}

\begin{definition}
Точки $(x,y)$ для которых верно $I(x,y) = 1$ будем называть заполненными.
\end{definition}

\begin{definition}
Точки $(x,y)$ для которых верно $I(x,y) = 0$ будем называть пустыми.
\end{definition}

Множество заполненных точек образует множество связных областей 
$$\{K_1, K_2 \dots K_n \},$$
таких что:
\begin{enumerate}
\item $\forall(x_1, y_1)\forall(x_2, y_2)(|x_1 - x_2| > 1 \wedge |y_1 - y_2|>1)$, где $(x_1, y_1)\in K_i$, $(x_2, y_2)\in K_j$ и $i\neq j$ 
\item $\forall(x_1, y_1)\exists(x_2, y_2)(|x_1 - x_2| \leq 1 \wedge |y_1 - y_2| \leq 1)$, где $(x_1, y_1),(x_2, y_2)\in K_i$ и $(x_1, y_1)\neq(x_2, y_2)$
\item $|K_i|\geq2$
\end{enumerate}

\begin{definition}
Всякую связную область $K_i$ будем называть контуром
\label{def:contour}
\end{definition}

\begin{definition}
Растровое изображение $I$ содержащие по крайней мере один контур будем называть растровым контурным изображением
\end{definition}

\begin{remark}
Не исключая общности, далее будут рассматриваться только растровые изображения содержащие один контур, а под растровым контурным изображением будет пониматься растровое изображение содержащие только один контур.
\end{remark}

\section{Модель контурного изображение}
В качестве математической модели представления растрового контурного изображения будем использовать четырех-основную алгебраическую систему вида.
\begin{definition}
Контурное изображение (далее, изображение) есть система вида
\begin{equation}
\mathfrak{M} = < A, R, V, M; Sector, Angle, Metric, Relation >
\label{eq:model}
\end{equation}
где
\begin{enumerate}
\item[] $A$ -- множество всевозможных дуг,
\item[] $R$ -- множество связей дуг,
\item[] $V \subset Z$ -- множество допустимых углов (например от 0 до 360 градусов),
\item[] $M \subset Z$ -- множество относительных мер,
\item[] $Sector: A \rightarrow V$ -- задает градусную меру дуги,
\item[] $Metric: A \rightarrow M$ -- функция сопоставляющая каждой дуге ее относительную величину,
\item[] $Angle: R \rightarrow V$ -- задает угол соединения двух дуг
\item[] $Relation: R \rightarrow A \times A$ -- сопоставляет каждой связи дуги, те дуги, которые она соединяет.
\end{enumerate}
\end{definition}

\begin{remark}
Важно отметить, что в этом представлении все множества являются конечными, и, если множества $A$ и $R$ являются фиктивными (чисто техническими элементами) данной модели, и определяются через функции, то множество $V = \{v_0, v_1, ..., v_n\}$ -- есть конечное множество чисел, с максимальным $v_{max}$ и минимальным $v_{min}$ элементами, разбитое шагом $\delta$ на $n+1$ элементов, где 

$$n = \frac{v_{max} - v_{min}}{\delta} $$
$$v_0 = v_{min}$$
$$v_{i} - v_{i-1} = \delta, \forall i=\bar{1,n} $$
$$v_n = v_{max} $$
\end{remark}

\begin{remark}
Как и множество $V$, множество $M$ конечно. Однако выбор верхней границы для $M$ не столь очевиден, так как не исключена возможность того, что разница в размерах между двумя дугами может быть весьма существенна (например в несколько миллионов раз). Однако, в рамках нашей области применения (распознавание символов), когда в качестве <<эталонного наблюдателя>> выступает человеческий глаз, разница что в 1000, что в 1000000 раз почти неразличима, и поэтому ею вполне можно пренебречь, выбрав в качестве максимального значения например 100000 процентов, а в качестве шага одну десятую процента. Таким образом всякая дуга может быть как больше так и меньше любой дуги не более чем в 1000 раз.
\end{remark}

Для наших целей важно всегда работать только с конечными множествами, что достигается рассмотрением конечных множеств $A$, $R$, а также предположением о наличии минимального шага возрастания количественных характеристик дуг и связей дуг.

Таким образом контурное изображение будет представляет собой систему дуг и связей дуг. Где всякая дуга определяется через ее градусную меру и через относительную (в данном контуре) длину дуги. А всякая связь определяется через угол связи и пару дуг которые она связывает.

\section{Преобразование растрового изображения}
Переход от растрового контурного изображения к изображению состоит из двух этапов. Первый этап — волновая скелетизация. С помощью скелетезации на основе растрового изображение строится граф (скелет), который визуально адекватно соответствует исходному изображению.

Пусть $I$ -- растровое контурное изображение и $K$ -- есть его контур.

\begin{definition}
Точку $q(x_1, y_1)$ будем называть соседом точки $p(x, y)$ если $|x-x_1| \le 1$ и $|y-y_1| \le 1$ и $p\neq q$.
Введем отношение соседства $N(p,q)$, которое истинно если $p$ сосед $q$.
\end{definition}

Очевидно что точка $p$ не может иметь более 8 соседей. Обозначим через $N_K^p$ множество всех соседей точки $p(x,y)$ лежащих контуре~$K$:
$$N_K^p = \{q | q\in K \wedge N(p,q)\}$$
Согласно определению контура (\ref{def:contour}) очевидно, что $K$ не имеет изолированных точек т.е.
$$\forall p \exists q: N(p,q)$$
$$p,q\in K$$

\begin{remark}
Скелетом $I$ будем называть граф $G(V,E)$, «интуитивно адекватно отражающий» исходное изображение.
\end{remark}

\begin{definition}
Волной $w$ будем называть конечное множество точек $\{p_j\}$.
\end{definition}

\begin{definition}
Множество волн $\{w_1,w_2,...,w_n\}$ будем называть подволнами волны $w$ если:
$$\bigcup\limits_{i=1,n}{w_i} = w$$
$$w_i\cap w_j = \emptyset,\; i,j = \overline{1,n},\; i\neq j;$$
и для любых двух точек $p\in w_i,\: g\in w_j$, где $i\neq j$ верно $\neg N(p,g)$.
\end{definition}

Введем функцию вычисляющую центр масс точек волны 
$$g(w)=\dfrac{\sum\limits_{p\in w} p}{\lvert w\rvert}$$

\subsection{Волновая скелетизация}
Опишем алгоритм построения скелета растрового контурного изображения.

%\begin{alg}
Зададим начальные условия. В качестве начальной волны подойдет любая точка из~$F$. 
Имеем следующую начальную конфигурацию:\\
${{w_0}^0} = \{p\}, p\in K$ -- начальная волна,\\
$W_0 = \{{w_0}^0\}$  -- множество волн,\\
$F_0 = K$ -- состояние заполненной области,\\
$G_0(V_0,E_0), V_0=\{p\}, E_0=\emptyset$ - начальное состояние скелета.\\

Определим $n$-ый шаг итерации следующим образом.
Для всякой $i$-ой волны ${w_i}^{k_i-1}$ из $W_{n-1}$ ($k_i$ - соответствует $k_i$-ой итерации $w_i$):
\begin{equation}
{w_i}^{k_i} = \bigcup_{p\in{w_i}^{k_i-1}}N_{F_{n-1}}^p\setminus\bigcup_{j<i}{w_j}^{k_j-1}
%\ilabel{skeleton1}
\end{equation}
Если $u_1,\ldots ,u_m$ есть подволны волны ${w_i}^{k_i}$, тогда
$$W^i_n=\{w_{l+1},\ldots ,w_{l+m}\}$$
где
$$w_{l+j}=u_j, j=\overline{1,m}$$
$$l = |W_{n-1}|+\sum\limits_{j<i}|W^j_n|.$$
Ребра в графе образуют вектора, связывающие центры масс полученных подволн с центром массы ${w_i}^{k_i-1}$
$$E^i_n=\{(v_{i}^{k_{i-1}}, v_{l+j}^{k_i})\}, j=\overline{1,m}$$
$$V^i_n=\{v_{l+j}^{k_i}\}, j=\overline{1,m}$$
$$v_i^j = g(w_i^j)$$
Если же волна $w_i^{k_i}$ не имеет разрывов и $w_i^{k_i}\neq\emptyset$, то
$$W^i_n=\{{w_i}^{k_i}\}$$
$$E^i_n=\{(v_{i}^{k_{i-1}}, v_{i}^{k_{i}})\}$$
$$V^i_n=\{v_{i}^{k_{i}}\}$$
Если $w_i^{k_i}=\emptyset$
$$W^i_n=\emptyset,E^i_n=\emptyset,V^i_n=\emptyset$$
Таким образом, при $s_n = |W_{n-1}|$:
$$W_{n} = \{W^i_n\}, i=\overline{1,s_n}$$
$$V_n = V_{n-1}\bigcup_i V^i_n, i=\overline{1,s_n}$$
$$E_n = E_{n-1}\bigcup_i{E^i_n}, i=\overline{1,s_n}$$
$$F_{n} = F_{n-1}\setminus P_{n-1},$$ 
$$P_{n-1} = \{p : p\in w, w \in W_{n-1}\}$$
Если $|F_{n}|=0$ то алгоритм прекращает цикл итераций, а граф
$$G = (V_n, E_n)$$
является скелетом исходного изображения $f$.\\

\begin{state} 
Алгоритм волновая скелетизация, остановится на любом контуре мощности $m$ удовлетворяющей условиям алгоритма.
\end{state}

\begin{proof}
Пусть $m=1$, тогда согласно алгоритму 
$$|F_1|=|F_0\setminus\{p:p\in w, w\in W_0\}| = |\{p\}\setminus\{p:p\in w_0\}|=|\{p\}\setminus\{p\}|=\emptyset$$
следовательно алгоритм прекращает свою работу а граф $G=(V_1,E_1)=(\{p\},\emptyset)$ является скелетом изображения.\\Пусть $m>2$ и существует такое $l$ что для всякого $k<l,|F_{k}| < |F_{k-1}|$ и $|F_{l}|=|F_{l-1}|$, тогда $P_{n-1}=\emptyset$, отсюда следует, что $\{p : p\in w, w \in W_{n-1}\}=\emptyset$, а это возможно только в двух случаях:\\
\begin{enumerate}
\item если $W_{n-1}$ пусто, тогда, в силу и в силу отсутствия изолированных точек, $F_{n-2}=\emptyset$, получаем противоречие с условием остановки.\\
\item если $\forall w\in W:|w|=0$, тогда, опять же в силу и в силу отсутствия изолированных точек, получаем что $F_{n-2}=\emptyset$, снова получаем противоречие с условием остановки.\\
\end{enumerate}
Следовательно такого $l$ не существует, и алгоритм сходится для всякой непустой заполненной области без изолированных точек.
\end{proof}
\begin{figure}[H]
%\includegraphics[width=\linewidth,keepaspectratio]{pic1_}
\caption{Схема разбиения графа «особыми» точками}
\end{figure}
%TODO картинка показывающая переход от монохромного изображения к графу

\subsection{Построение модели  $\mathfrak{M}$ для графа $G$}
Прежде чем дать форммальное описание алгоритма, рассмотрим как он работает неформально:

\begin{enumerate}
\item Для каждого простого пути выполняется:
	\begin{enumerate}
		\item Разбиение пути по точкам смены направления обхода
		\item Для каждого разбиения выполнятся
		\begin{enumerate}
			\item Разбиение спиралей. Чтобы определить закручен ли путь по спирали, надо проверить пересекает ли хорда путь. Если пересечение есть, то необходимо разбить путь точками пересечения
			\item Для каждого разбиения выполняется:
			\begin{enumerate}
				\item Разбиение по точкам перегиба. Точками перегиба считаются образующие две дуги отклонившиеся от угла идеального соединения. Угол $\gamma$ идеального соединения двух дуг градусной меры $\alpha$ и $\beta$: $$\gamma=2\pi-(\alpha+\beta)$$
			\end{enumerate}
		\end{enumerate}		
	\end{enumerate}	
\item Результатом п.1 является множество подпутей, каждый из которых переводится в дугу. Градусная мера дуги вычисляется с использование формулы Гюйгенса. Определив наиболее удаленную точку пути от хорды стягивающей путь, и вычислив её положение относительно хорды направленной от начала к концу пути, мы определяем направление обхода. Если точка слева, то обход ведется по часовой стрелке, если точка справа — обход ведется против часовой стрелки, если же точка лежит на прямой, то верны оба утверждения.
\item Расчет связей дуг. Связь между двумя дугами существует, если пути образующие дуги имели общие вершины. Угол соединения между дугами рассчитывается, как угол между стягивающими их хордами
\end{enumerate}

\begin{remark}
Стоит отметить что разбиение на дуги, как правило, выполняется на интерполированном графе в котором часть узлов 
удаленно в силу их избыточности. 
\end{remark}

\begin{remark}
Хотя разбиение и может быть использовано как есть, на практике полезнее добавлять возможность вариации параметров, например, допускать возможность некоторого отклонения от угла идеального соединения или для точек смены направления расширять область проверки на смену направления обхода.
\end{remark}

Пусть $\{P_1, P_2, \dots, P_m\}$ -- множество простых цепей графа $G=(V,E)$ таких что:
\begin{enumerate}
\item $P_i = v^i_1, v^i_2, \dots v^i_{n_i}$
\item $\forall v\in P_i \forall u \in P_j(v\neq u)$, где $i\neq j$
\item $d(v^i_1)\neq 2$ и $d(v^i_{n_i})\neq 2$
\item $\forall j\notin\{1, n_i\}\left[d(v^i_j) = 2\right]$
\item $\bigcup_i P_i = V$
\end{enumerate}
%TODO добавить частный случай о графе образованного одним циклом (буква О)

\begin{definition}
Пусть $r(p,v)$ -- есть растояние от точки $p$ до вектора $v$ со знаком
\end{definition}


\begin{definition}
Будем говорить что в узле $v_i$ меняется направление обхода простого  
$$\{v_1,...,v_{i-1},v_i,v_{i+1},...,v_n\}$$
в окрестности $\epsilon\in \{2, 3, ...\}$, если
$$
Sign(\sum\limits_{j=i-\epsilon}{r(v_j,h)}) \neq Sign(\sum\limits_{j=i+\epsilon}{r(v_j,h)})
$$
где $h = (v_{i-\epsilon}, v_{i+\epsilon})$

Если соотношение выполняется для $\epsilon=2$ будем просто говорить, что в точке $v_i$ меняется направление обхода.
\end{definition}

Пусть множество точек  $\{ v_{j^i_{1}}, \dots v_{j^i_{k_i}} \}$ множество точек смены направления обхода цепи $P_i$.
Определим множество индексов задающих разбиение пути $P_i = v_1, \dots v_n$ по направлению обхода
$$
I^{dev}_{P_i} = \{ j^i_{1}, \dots j^i_{k_i} \} \cup \{1, n\}
$$
$$
j^i_{k} < j^i_{k+1}
$$
Тогда следующие семейство множеств узлов определяют разбиение пути $P_i$ на подпути:
$$
S^{dev}_{P_i} = \bigcup_{l \in I^{dev}_{P_i}}
	\{
		\{ v_t\;|\;t\in \{j\;|\;i_l\leq j\leq i_{l+1}\}\}
	\}
$$
Каждому множеству узлов можно однозначно сопоставить подпуть $P_i$

\begin{definition}
Определим множество индексов узлов задающие разбиение подпути $P = v_1, \dots v_n$ закрученного в спираль следующим образом
$$I^{spir}_P = \left\{ \min_{|r(h, v_i)|}(i, i+1)\;|\;Sign(r(h,v_i)) \neq Sign(r(h,v_{i+1})) \wedge i=1,n-1 \right\}
$$
$$
h = (v_1, v_n)
$$
%TODO картинка цепи закрученной в спираль
\end{definition}

Формула Гюгейнса позволяет с высокой точностью вычислить градусную меру дуги

\begin{definition}
Пусть задан простой путь $P = v_1, \dots v_n$, тогда будем говорить что дуга $a\in A$ соответствует пути $P$ если выполняются следующие условия:
\begin{enumerate}
\item $Sector(a) =$ 
\end{enumerate}
\end{definition}

\begin{definition}
Определим множество индексов узлов задающие разбиение подпути $P = v_1, \dots v_n$ по резкости угла перегиба.

\end{definition}

\begin{definition}
Разбиением пути $P$ на характеристические подпути будем считать разбиение определенное следующим образом:

$$
S_{P_i} =
 	\bigcup_{ j\in I^{spir}_P,\;P\in S^{dev}_{P_i}} 
 	\left\{
 	 	\left\{
 	 		 v_t\;|\;t\in\{j\;|\;i_l\leq j\leq i_{l+1} \} 
 	 	\right\}	 
 	 \right\}
$$
\end{definition}

\begin{definition}
Пусть $P = v_1, \dots v_n$ простой путь тогда:
\begin{enumerate}
\item $f_{begin}(P) = v_1$ -- начало пути
\item $f_{end}(P) = v_n$ -- конец пути
\item $h_P = v_n - v_1$ -- вектор хорды стягивающей путь
\end{enumerate}
\end{definition}

\begin{definition}
Пусть задано разбиение графа $G=(V,E)$ на характеристические подпути $S=\{P_1, \dots P_m\}$, где $P_i=v^i_1, \dots v^i_{n_i}$, будем говорить что алгебраическая система
$$\mathfrak{M} = < A, R, V, M; Sector, Angle, Metric, Relation >$$
соответствует графу $G$ если выполняются следующие условия:
\begin{enumerate}
\item $Sector(a_i) = 2l + \frac{1}{3}(2l - L)$, где $l=|v^i_1-v^i_m|$, $i_m = \frac{n_i}{2}$,  а $L=|h_{P_i}|$
\item $Metric(a_i) = \frac{a_i}{a_1}$
\item $Relation(r_{ij}) = (a_i, a_j)$, где $i\neq j$ и неопределенно в противном случае
\item $Angle(r_{ij})=
	\left\{		
	\begin{array}{ll}
	\hat{(-h_{P_i}, -h_{P_j}}) & f_{begin}(P_i) = f_{end}(P_j) \\
	\hat{(-h_{P_i}, h_{P_j}}) & f_{begin}(P_i) = f_{begin}(P_j) \\
	\hat{(h_{P_i}, h_{P_j}}) & f_{end}(P_i) = f_{begin}(P_j) \\
	\hat{(h_{P_i}, -h_{P_j}}) & f_{end}(P_i) = f_{end}(P_j) \\	
	\text{неопределенно}, & \text{если } r_{ij} \text{ -- неопределенно}
	\end{array}
	\right\}	$	
\end{enumerate}
\end{definition}

\section{Постановка задачи распознавания}

\section{Интерпретация}

\section{Оценка сложности}

%\begin{definition}
%Определим множество точек разбиение подпути $P = v_1, \dots v_n$ по углу сгиба в окрестности $\epsilon\in Z^+$
%$$V_{spiral}^P = \left\{ \min_{|r(h, v)|}(v_i, v_i+1)\;|\;Sign(r(h,v_i)) \neq Sign(r(h,v_{i+1})) \wedge i=1,n-1 \right\}$$
%%TODO картинка цепи с острым углом
%\end{definition}

%TODO картинка показывающая переход от графа к дугам и связям дуг

%\end{alg}
%\chapter{Оформление различных элементов} \label{chapt1}
%
%\section{Форматирование текста} \label{sect1_1}
%
%Мы можем сделать \textbf{жирный текст} и \textit{курсив}.
%
%%\newpage
%%============================================================================================================================
%
%\section{Ссылки} \label{sect1_2}
%Сошлёмся на библиографию: \cite{bib1}, \cite{bib2}, \cite{bib3,bib4,bib5}.
%
%Сошлёмся на приложения: Приложение \ref{AppendixA}, Приложение \ref{AppendixB2}.
%
%Сошлёмся на формулу: формула (\ref{eq:equation1}).
%
%Сошлёмся на изображение: рисунок \ref{img:knuth}.
%
%%\newpage
%%============================================================================================================================
%
%\section{Формулы} \label{sect1_3}
%
%\subsection{Ненумерованные одиночные формулы} \label{subsect1_3_1}
%
%Вот так может выглядеть формула, которую необходимо вставить в строку по тексту: $x \approx \sin x$ при $x \to 0$.
%
%А вот так выглядит ненумерованая отдельностоящая формула c подстрочными и надстрочными индексами:
%$$
%(x_1+x_2)^2 = x_1^2 + 2 x_1 x_2 + x_2^2
%$$
%
%При использовании дробей формулы могут получаться очень высокие:
%$$
%  \frac{1}{\sqrt(2)+
%  \displaystyle\frac{1}{\sqrt{2}+
%  \displaystyle\frac{1}{\sqrt{2}+\cdots}}}
%$$
%
%В формулах можно использовать греческие буквы:
%$$
%\alpha\beta\gamma\delta\epsilon\varepsilon\zeta\eta\theta\vartheta\iota\kappa\lambda\\mu\nu\xi\pi\varpi\rho\varrho\sigma\varsigma\tau\upsilon\phi\varphi\chi\psi\omega\Gamma\Delta\Theta\Lambda\Xi\Pi\Sigma\Upsilon\Phi\Psi\Omega
%$$
%
%%\newpage
%%============================================================================================================================
%
%\subsection{Ненумерованные многострочные формулы} \label{subsect1_3_2}
%
%Вот так можно написать две формулы, не нумеруя их, чтобы знаки равно были строго друг под другом:
%\begin{eqnarray}
%  f_W & = & \min \left( 1, \max \left( 0, \frac{W_{soil} / W_{max}}{W_{crit}} \right)  \right), \nonumber \\
%  f_T & = & \min \left( 1, \max \left( 0, \frac{T_s / T_{melt}}{T_{crit}} \right)  \right), \nonumber
%\end{eqnarray}
%
%Можно использовать разные математические алфавиты:
%\begin{eqnarray}
%\mathcal{ABCDEFGHIJKLMNOPQRSTUVWXYZ} \nonumber \\
%\mathfrak{ABCDEFGHIJKLMNOPQRSTUVWXYZ} \nonumber \\
%\mathbb{ABCDEFGHIJKLMNOPQRSTUVWXYZ} \nonumber
%\end{eqnarray}
%
%Посмотрим на систему уравнений на примере аттрактора Лоренца:
%
%$$
%\left\{
%  \begin{array}{rl}
%    \dot x = & \sigma (y-x) \\
%    \dot y = & x (r - z) - y \\
%    \dot z = & xy - bz
%  \end{array}
%\right.
%$$
%
%А для вёрстки матриц удобно использовать многоточия:
%$$
%\left(
%  \begin{array}{ccc}
%  	a_{11} & \ldots & a_{1n} \\
%  	\vdots & \ddots & \vdots \\
%  	a_{n1} & \ldots & a_{nn} \\
%  \end{array}
%\right)
%$$
%
%
%%\newpage
%%============================================================================================================================
%\subsection{Нумерованные формулы} \label{subsect1_3_3}
%
%А вот так пишется нумерованая формула:
%\begin{equation}
%  \label{eq:equation1}
%  e = \lim_{n \to \infty} \left( 1+\frac{1}{n} \right) ^n
%\end{equation}
%
%Нумерованых формул может быть несколько:
%\begin{equation}
%  \label{eq:equation2}
%  \lim_{n \to \infty} \sum_{k=1}^n \frac{1}{k^2} = \frac{\pi^2}{6}
%\end{equation}
%
%В последствии на формулы (\ref{eq:equation1}) и (\ref{eq:equation2}) можно ссылаться.
%
%%\newpage
%%============================================================================================================================
%
%\clearpage
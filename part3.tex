\chapter{Применение} \label{chapt3}
\section{Распознавание символов}
Наиболее естественным применением разработанного подхода является, очевидно, распознавание символов (букв, логотипов и т.п.).

Рассмотрим метод на примере анализа буквы <<a>>

...

Отдельно стоит упомянуть задачу поиска партизана в лесу

...
\section{Оценка устойчивости битумных эмульсий}
В настоящее время трудно назвать область строительства, где бы не применялись эмульсии. Они используются в дорожном и в гражданском строительстве в качестве связующих с различными наполнителями, а также в качестве гидроизоляционных и лакокрасочных материалов. При любых технологиях использования эмульсий мы сталкиваются с одними и теми же проблемами, касающимися подбора состава, приготовления, определения физико-механических характеристик, стабильности, контроля распада эмульсий и получения продукции с необходимыми свойствами [1]. Далее мы будем рассматривать только прямые битумные и битумно-латексные эмульсии, которые являются наиболее крупнотоннажным продуктом: мировое использование составляет миллионы тонн в год

Традиционные методы оценки свойств битумных эмульсий включают: определение содержания вяжущего с эмульгатором, определение устойчивости эмульсии при перемешивании, определение остатка на сите, определение условной вязкости, определение устойчивости при хранении, определение адгезии эмульсий с поверхностью наполнителей, определение устойчивости при транспортировки и т.п. [2].  Наряду с традиционными методами изучения  качества эмульсии, во многих приложениях желательно знать более тонкие характеристики, например: функцию распределения по размерам. Эта характеристика является одним из важнейших параметров и позволяет предсказывать большинство свойств эмульсии. Обычноразмер частиц оценивают с помощью определения остатка на сите с заданным размером ячейки, но такой метод позволяет оценивать только верхний предел размеров частиц эмульсии. Полная картина распределения частиц по размеру может быть измерена с использованием таких технических приёмов как рассеяние света, микроскопия с анализом изображений, или же с помощью техники электроозонирования («техники Культера» - Coulter). Точный анализ размеров частиц битумной эмульсии может решить многие проблемы, которые в настоящее время являются актуальными в сфере производства битумных эмульсий:
\begin{enumerate}
\item Влияние эмульгатора и его концентрации на размер битумных частиц эмульсии.
\item Влияние модифицирующих битум добавок на качество получаемой эмульсии.
\item Корректировка технологической схемы производства эмульсии.
\item Влияние размера битумных частиц на основные физические свойства эмульсии.
\end{enumerate}

Оптическая микроскопия, как способ распределения частиц по размерам, является наиболее удобным и точным.Например, если в способе «рассеяние света» могут возникнуть проблемы с отражением света от черных поверхностей, какими являются частицы битума, то в способе микроскопии, при высоком контрасте черного цвета, напротив, можно отличить частицы от среды, в которой они находятся.

%TODO вставить изображение частиц до обратботки / после обработки
Очевидно изображения такого вида являются одним из примеров растровых контурных изображений, с несколькими контурами. В отличие от задачи распознавания символов, здесь нет необходимости анализировать скелет изображения. Куда более важную роль играет внешний контур. Разбивая контур на дуги мы можем классифицировать частицы по по уровню распада:
\begin{enumerate}
\item одиночные
\item слипшиеся
\item распавшиеся
\end{enumerate}
Большое количество распавшихся частиц является свидетельством того что смесь является неустойчивой, а следовательно некачественной.
...

\section{Автоматизация составления ПОДД}
%
%\section{Таблица обыкновенная} \label{sect3_1}
%
%Так размещается таблица:
%
%\begin{table} [htbp]
%  \centering
%  \parbox{15cm}{\caption{Название таблицы}\label{Ts0Sib}}
%%  \begin{center}
%  \begin{tabular}{| p{3cm} || p{3cm} | p{3cm} | p{4cm}l |}
%  \hline
%  \hline
%  Месяц   & \centering $T_{min}$, К & \centering $T_{max}$, К &\centering  $(T_{max} - T_{min})$, К & \\
%  \hline
%  Декабрь &\centering  253.575   &\centering  257.778    &\centering      4.203  &   \\
%  Январь  &\centering  262.431   &\centering  263.214    &\centering      0.783  &   \\
%  Февраль &\centering  261.184   &\centering  260.381    &\centering     $-$0.803  &   \\
%  \hline
%  \hline
%  \end{tabular}
%%  \end{center}
%\end{table}
%
%%\newpage
%%============================================================================================================================
%
%\section{Параграф - два} \label{sect3_2}
%
%Некоторый текст.
%
%%\newpage
%%============================================================================================================================
%
%\section{Параграф с подпараграфами} \label{sect3_3}
%
%\subsection{Подпараграф - один} \label{subsect3_3_1}
%
%Некоторый текст.
%
%\subsection{Подпараграф - два} \label{subsect3_3_2}
%
%Некоторый текст.
%
%\clearpage